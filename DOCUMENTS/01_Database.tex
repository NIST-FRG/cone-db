% !TEX root = ConeDB_Technote.tex
\section{Organization of the NIST Cone Calorimeter Flammability Database} \label{sec:DB-overview}
Key components:
\begin{itemize}
\item The NIST Cone Calorimeter Github Repository
\item Standard formatting and naming convention (files, folders, data)
\item Metadata (test and material)
\item Derived properties/model inputs
\item Graphical User Interface (GUI https://www.nist.gov/)
\item Data Sources (Legacy and New; references)
\end{itemize}

\subsection{The NIST Cone Calorimeter Github Repository} \label{ssec:DB-Github}

\subsection{File and Folder Naming and Formatting} \label{ssec:DB-formatting}

\subsection{Metadata (test and material)} \label{ssec:DB-metadata}
Measurement data from thousands of experiments on a wide range of materials including commodity plastics, woods, cables, vegetative fuels, and flooring and other building materials is available on the NIST Cone Calorimeter database. A continuously-updated, version controlled, searchable `Table of Materials' is available online: **link.
Table of Properties
\subsection{Derived properties/model inputs} \label{ssec:DB-properties}
Derived property values from each of the materials included in the NIST Cone Calorimetry Database (including commodity plastics, woods, cables, vegetative fuels, and flooring and other building materials) is available online: **link . This `Table of Properties' is continuously-updated, version controlled, and searchable.

\subsection{Graphical User Interface} \label{ssec:DB-GUI}


\subsection{Data Sources (NIST Legacy and New; external; references)} \label{ssec:DB-Data}


\begin{figure}[h] 
	\centering 	\includegraphics[width=6in]{../FIGURES/llama1.png}
	\caption{Overview of the Material Flammability Database}
	\label{fig:matl-fl-db-flowchart}
\end{figure}



\subsubsection{Data for Property Calibration}
\subsubsection{Flammability data Automated Calibration Tools (FACT)}
Flammability data Automated Calibration Tools (FACT) is a suite of tools and data in support of predictive fire modeling. The tools within FACT can rapidly determine material parameter inputs needed for making robust and accurate computational predictions of fire growth for a broad range of materials. Such predictions can support the development of reduced flammability materials and a safer built environment.

Note that FACT is still in development. Currently, tools exist for:
\begin{itemize}
\item Extracting pyrolysis kinetic parameters from thermogravimetric analysis (TGA) data,
\item Extracting heats of combustion from microcombustion calorimetry (MCC) data, and
\item Extracting specific heat capacities and heats of pyrolysis from differential scanning calorimetry (DSC) data.
\end {itemize}

Details of the FACT suite of tools are provided in Volume 2 of this report~\cite{MFDBv2FACT} and related publications~\cite{bruns2021automated}.
\subsubsection{Material Properties // Fire Model Inputs}
\subsubsection{Data for Fire Model Validation}
\subsubsection{Guided Uncertainty Reduction Utilities (GURU)}
The Guided Uncertainty Reduction Utilities (GURU) suite of computational tools will provide fire scientists and
engineers with detailed uncertainty characterizations to guide the integrated development of material flammability measurement science and predictive fire modeling. Such guidance will lead to cost-effective and efficient improvement of reliable fire growth predictions to help engineer reduced flammability materials that will protect people, property, and the environment from the hazard of fire.

\subsection{How to Access and Interact with the NIST Material Flammability Database} \label{sec:MFDB-user-interface}
\subsubsection{The Github Repository}
\subsubsection{The GUI/Website}
Table of Materials\\
Table of Properties\\


\subsection{User Support for the NIST Material Flammability Database} \label{ssec:MFDB-support}
\subsubsection{The Version Number}
\subsubsection{Support Requests and Bug Tracking}
